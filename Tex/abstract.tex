


The program used in this project can be found at \href{https://github.com/kjetka/Project3}{Github}. 

  Egen klasse for å kjøre algoritmer i (tidsteg, skrive fil) 
     
PLAN:  
Torsdag: 
  Kjetil: Teori, metode, (intro?), Fikse angulærmoment 
  Vilde: Plots, andre resultater (CPU-tid...),  (skrive resultater)



Test: 		
Energy conservation, modulus "position" (lengde vektor) bevart
				Alle vectorer samme str.



	Printe + plotte energi stabilitet mellom euler og verlet.


To do:
	OBS Unit tests
	
	3b:
								Forklare objektorientering, hvorfor deler kan generaliseres. 
	
	3c: 					  
	- Find out which initial velocity that gives a circular motion (plot)
	- Test stability (energy-stability) as function of dt (both Verlet and Euler)
	- Plot the earth orbiting the sun
	- Check (for the circular orbit) that the energy is conserved (plot - both kin and pot separated and together?)
	- Check that angular moment is conserved
	
	- Discuss differences between Euler and Verlet
		- number of FLOPS + CPU time
	
	
 * Plotte ulike dt-er
 * Plott energi som funk av ulike dt
 * Referere til convergens - funksjonen.
 * Vise at angulærmoment bevart

	3d: 	
	- Find escape velocity (plot)
	 	- Compare with numerical results(Result? or Discussion?)
	- Find exact escape velocity (theory?)
	- Changing beta (plot)
		- Comment result + What happens when beta -> 3 ?
	
					  Exact løsning escape vel
								Plots ulike init.hastigheter
								Bytte gravitasjonskrefter... 
								
	3e:
	- How much does Jupiter alter Earth's orbit?
	- Position of Jupiter and Earth (plot)	
	- Plot Earth's motion for increased mass of Jupiter (3 masses)
	- 
	- Discuss stability of velocity verlet (3 body)

 * 3 ulike masser
 * Plotte alle banene
 * Stabilitet: Energi-plot

	3f:
	- Find center off mass - use as origin
	- Give sun initial velocity so momentum is zero (origin is fixed)
	- Compare with 3e)
	- Extend to all planets (plot)
	- Discuss difference 3e) and 3f) (3 body)
	- Discuss result of all planets
	
	3g:
	- 	Find perihelion for both relativistic and non-relativisitc (table)
	- Relativistic - should be a few magnitudes smaller.	
	-  Can the observed perihelion
precession of Mercury be explained by the general theory of relativity?
	
		
	FLOPS euler/Verlet
Result:		
	- Find out which initial velocity that gives a circular motion (plot)
	- Test stability (energy-stability) as function of dt (both Verlet and Euler)
	- Plot the earth orbiting the sun
	- Check (for the circular orbit) that the energy is conserved (plot - both kin and pot separated and together?)
	- Check that angular moment is conserved

	- Find escape velocity (plot)
	 	- Compare exact and numerical results
	- Find exact escape velocity (theory?)
	- Changing beta (plot)
		- Comment result + What happens when beta -> 3 ? (last part in discussion?
	- How much does Jupiter alter Earth's orbit?
	- Position of Jupiter and Earth (plot)	
	- Plot Earth's motion for increased mass of Jupiter (3 masses)
	- Find center off mass - use as origin
	- Give sun initial velocity so momentum is zero (origin is fixed)
	- Compare with 3e)
	- Extend to all planets (plot)
	- Find perihelion for both relativistic and non-relativisitc (table)
	- Relativistic - should be a few magnitudes smaller.	
	-  Can the observed perihelion
precession of Mercury be explained by the general theory of relativity?

Discussion:
	- Discuss differences between Euler and Verlet
		- number of FLOPS + CPU time
	- Discuss stability of velocity verlet (3 body)
	- Discuss difference 3e) and 3f) (3 body)
	- Discuss result of all planets

