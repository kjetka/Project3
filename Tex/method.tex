
In this project we tested two numerical integration techniques, the Euler method and the velocity Verlet method.  This section is largely based on Hjort-Jensen \cite{Jensen}	



\subsection{The Euler  method}


When evaluating the function $ x(t)$ in the interval between $ t $ and $ t+h $ it is natural to do a Taylor expansion:  

\begin{equation}
x(t+h) = x(t) +\sum\limits_{n=1}^{\infty} 	\frac{ h^n}{n!}\dv[n]{x(t)}{t}   \label{eq:Taylor}
\end{equation}

By choosing a small $ h $, it is sufficient to truncate the sum after n=1, giving  $ x(t+h) = x(t) + h \dv{x}{t} + O(h^2) $. The term$ O(h^2) $ contains the rest of  the infinite sum, often called the truncation error. In order for a computer to use this method it is necessary to discretise  the expression, substituting $ x(t) \rightarrow x( t_i ) \rightarrow x_i$. The Euler method can thus be expressed as 

\begin{equation}
	x_{i+1} = x_i + h\dot{x}_i + O(h^2)
\end{equation}
 
 Combining this with equations  \ref{eq:vx} and \ref{eq:x} we get an algorithm for doing a  1 dimentional integration for the solar system:
 
 \begin{align}
 a_i &= F(t_i)\\
 v_{i+1} &= v_i + ha_i\\
  x_{i+1} &= x_i + hx_i
 \end{align}
 
\textbf{ This gives in total 4 floating points operations per iteration, per dimension. For our two dimentional system this results in a total of 8N FLOPS per iteration, with $ N = \frac{1}{h} $.}
 
 


\subsection{The Velocity Verlet method}

The velocity Verlet can be derived from the same Taylor expansion as the Euler method, equation \ref{eq:Taylor}. As with the Euler method, we will truncate the taylor series for position after n=1. However, for the velocity there exist a similar Taylor series: $  v_{i+1} = v_i + h\dot{v}_i + \frac{h}{2} \ddot{v}_i + O(h^3)$. Unfortunately, there is no expression for $ \ddot{v}_i $, but it can be approximated by $ h\ddot{v}_i \simeq \dot{v}_{i+1}-\dot{v}_i$. Adding this to the expression for $ v_{i+1} $ and doing some simple algebra one get the final expression for the velocity:

\begin{equation}
v_{i+1} = v_i +  \frac{h}{2} \left(\dot{v}_{i+1}+\dot{v}_i\right)+ O(h^3) \label{eq:vel_verlet}
\end{equation}

Looking a bit closer at equation \ref{eq:vel_verlet}, there is a immediate problem. In order to calculate $ v_{i+1} $ one need to already know $ v_{i+1} $. This problem can be solved by updating the velocity in two steps, which gives the following algorithm:

\begin{align}
v_{i+\frac{1}{2}} &= v_i + \frac{h}{2} a_i\\
x_{i+1} &= x_i + v_{i+\frac{1}{2}}\\
a_{i+1} &= \frac{F_{i+1}}{m}\\
v_{i+1} &=v_{i+\frac{1}{2}}. + \frac{h}{2} a_{i+1}\\
\end{align} 

The numerical cost of the algorithm can be understood when analysing the number of floating points operations. In the case of the velocity verlet method there are 8 FLOPS per iteration and dimension. As the value $ \frac{h}{2} $ a constant, it can be calculated in advance of the loop, thus reducing the FLOPS to 6. This gives a total of 12 FLOPS per iteration for the two dimensional system. 






\subsection{Object orientation}
In order to simplify the calculation of an ensemble of planets, each with velocities, positions, energies and angular momentum, it is useful to generalize the code in an object oriented way. We chose to create one class for the planets, where all the internal dynamics (energies, position, velocity, ...) were stored. The forces experienced by the planets are specific for this project and we kept this in the Planet class, so that the second class, the Solver class, could be more general and easier reused.

This Solver class is were all the technicalities are located, including the different integration methods. For each time step one need the location of every planet in the system and Solver includes therefore a function that for each time step loop over all the planets. In order to update their position it is necessary to again loop over all the different planets in order to find the total gravitational force exerted on the current planet.  

When all the calculations are taken care of by the solver class and all the properties of the different planets are stored in each planet object. This means that the main part of the program only needs to initialize the different planetary instances , adding these to the a instance of the Solver class which is initialized according to what output we want to achieve, see the snippet below for how this looks in 'main'. 
 
\begin{lstlisting}[language=C++]
	Planet earth("Initializing inputs");
		Planet sun("Initializing inputs");
	Solver verlet("Initializing inputs");
	verlet.add(earth);
	verlet.add(sun);
	verlet.add(mercury);
	verlet.algorithm("input variables");
\end{lstlisting}



\subsection{Choice of origin}

The choice of origin is important, as it effects the physics of the system. As the sun is massively more massive than the earth and any other object in the solar system, it would be a good approximation to have the sun fixed in the origin. For the two-body simulations we chose to define the initial position of the sun as the origin instead, allowing the sun to move freely. However, this opens up for the sun to move away from the origin which affects the orbital of the earth. 

For the simulations of the three-body solar system it was necessary too choose the barycentre of the universe as origin. This position, $ \vec{R} $ is chosen so that the mass-weighted position of all the planets relative to R is zero:

\begin{equation}
	\sum\limits_{i=i}^{N} m_i(r_i-R) = 0 \label{eq:barycentre}
\end{equation}

In addition the sun had to be initialized with velocities so that the total momentum of the system was $\vec{p}_{system}= 0 $, or else the system would slowly move further and further away from origin. Thus the initial velocity of the sun must equal $ \vec{v}_{sun} = \frac{1}{M} \sum\limits_{i=0}^{N} m_iv_i $ for N planets in the sun-planet system and a total mass of the system equal to $ M $. With this choice of initial velocities the barycentre will be in the origin throughout the simulations, given that the momentum is conserved numerically. 

When simulating the full solar system we used positions and velocities given by NASA \cite{nasa}, with the barycentre as origin. 
